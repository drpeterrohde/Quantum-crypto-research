\documentclass[twocolumn, aps, amsmath, amssymb, nofootinbib, superscriptaddress, longbibliography, doublefloatfix, table-of-contents, eqsecnum, rmp]{revtex4-2}

\usepackage[pdftex]{graphicx}
\usepackage{mathrsfs}
\usepackage[colorlinks, breaklinks, urlcolor={blue}, linkcolor={red}, citecolor={blue}]{hyperref}
\usepackage{amsmath}
\usepackage[english]{babel}
\usepackage{booktabs}
\usepackage{amssymb}
\usepackage{type1cm}
\usepackage{caption}
\usepackage{url}
\usepackage[breaklinks]{hyperref}
    
\frenchspacing
    
\captionsetup[figure]{margin=0pt, font=small, labelfont=bf, labelsep=endash, justification=centerlast, labelsep=colon}
\captionsetup[algorithm]{margin=0pt, font=small, labelfont=bf, labelsep=endash, justification=centerlast, labelsep=colon}
    
\begin{document}

\title{Graph state digital signature scheme}

\begin{abstract}
\end{abstract}

\maketitle

\tableofcontents

\section{Graph states}

* Definition via stabilisers.

* Transformation and measurement rules.

\section{Graph states signature scheme}

* Describe one-way interpretation: $G\to\rho(G)$.

* Describe commit-reveal interpretation.

* Random graph construction.

* Describe protocol.

* How encoding via pairwise measurements works.

* Local correction rules.

\section{Security proof}

* Proof that adjacency matrix exponentially converges to uniform and separable random edge set.

* Argue that this affords information theoretic security, parameterised by statistical security $\varepsilon=1/2^n$.

\end{document}
