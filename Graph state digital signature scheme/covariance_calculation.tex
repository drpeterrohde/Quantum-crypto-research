\documentclass{article}
\usepackage[utf8]{inputenc}
\usepackage{amssymb}
\usepackage{amsmath}
\usepackage{graphicx}
\usepackage{mathtools}
\usepackage{bbm}
\graphicspath{{./images/}}
\usepackage[margin = 0.5in]{geometry}
\usepackage{mathtools}
\usepackage{pst-node}
\usepackage{tikz-cd}
\usepackage{tikz}
\usetikzlibrary{automata,positioning}
\usepackage{textcomp}
\usepackage{amsthm}


\title{Caclulation}
\date{}

\begin{document}

\maketitle

\section{Calculation}
We calculate what the covariance matrix looks like after a single partition/local complementation operation. We assume that the partition is sampled according to the stochastic block model. So the graph is partitioned in to subgraphs $V_1$ and $V_2$ and an edge between two vertices in $V_i$ exists with probability $p$, while an edge between a vertex in $V_1$ and a vertex in $V_2$ exists with probability $q$. We assume $|V_1| = |V_2|$.  
\subsection{Edge Exists}
First calculate $P(\text{edge e exists after partition/complement operation})$. Let $e$ connect vertices $i$ and $j$. This equals $$P(\text{e exists after complementing}|(i \text{ and }j \in V_1))\cdot P(i \text{ and }j \in V_1)$$ $$+ P(\text{e exists after complementing}|(i \text{ and }j \in V_2))\cdot P(i \text{ and }j \in V_2)$$ $$+ P(e \text{ exists after complementing }|i \in V_1, j \in V_2) \cdot P(i \in V_1, j \in V_2)$$ $$+ P(e \text{ exists after complementing }|i \in V_2, j \in V_1) \cdot P(i \in V_2, j\in V_1).$$ When vertices $i$ and $j$ are in the same partition after a complement, the probability the edge exists after the complement is $1-p$. Complements don't affect edges which go across partitions, so the probability of that edge existing after the complement is still $q$. So the above equals $$(1-p)\cdot \frac{1}{4} + (1-p)\cdot{1}{4} + q\cdot \frac{1}{4} + q\cdot \frac{1}{4}.$$ $$= \frac{1 - p + q}{2}.$$
\subsection{Variance}
Let $X_{e_i}$ denote the random variable where $X_{e_i} = 1$ if the edge $e_i$ exists and $0$ otherwise. So the covariance matrix we want to calculate looks like $$\begin{bmatrix}
    Var(X_{e_1}) & & & Cov(X_{e_1}, X_{e_{n^2}})\\
    Cov(X_{e_2}, X_{e_1}) & Var(X_{e_2}) & & & \\
    . & . & . \\
    \\
    \\
    Cov(X_{e_{n^2}}, X_{e_1}) & & & Var(X_{e_{n^2}})\\
\end{bmatrix}$$
We first calculate the diagonal elements (i.e. the variances). By definition, $Var(X_{e_i}) = E(X_{e_i}^2) - E(X_{e_i})^2$. $E(X_{e_i})$ is just the probability that the edge $e_i$ exists, which we already calculated. 
\\
\\
Consider the random variable $Y_{e_i} = x_{e_i}^2$. Then $Y_{e_i}$ is $1$ when the edge $e_i$ exists after the complement and zero otherwise. That is, it has the exact same distribution as $X_{e_i}$. Therefore, $$Var(X_{e_i}) = E(Y_i) - E(X_{e_i})^2$$ $$= \left( \frac{1 - p + q}{2}\right) - \left(\frac{(1-p+q)^2}{4}\right).$$ 
\subsection{Covariance}
We calculate $Cov(X_{e_i}, X_{e_j})$. By definition, $$Cov(X_{e_i}, X_{e_j}) = E(X_{e_i} - \mu_{X_{e_i}})(X_{e_j} - \mu_{X_{e_j}})$$ $$= E(X_{e_i}X_{e_j} - X_{e_i}\mu_{X_{e_j}} - X_{e_j}\mu_{X_{e_i}} + \mu_{X_{e_i}}\mu_{X_{e_j}}).$$ We just need to calculate the product term. Then we can use the linearity to get the whole thing. See that $X_{e_i}X_{e_j}$ is the random variable which takes the value $1$ if both edges $e_i$ and $e_j$ exist after complementing and zero otherwise. So $E(X_{e_i}X_{e_j}) = P(X_{e_i} = 1 \text{ and }X_{e_j} = 1)$. Let $e_i$ connect vertices $i,j$ and $e_j$ connect vertices $(i',j')$. This is $$P(X_{e_i = 1} \land X_{e_j = 1} | (i,j), (i',j') \in V_1) \cdot P((i,j),(i',j') \in V_1)$$ $$+ P(X_{e_i = 1} \land X_{e_j = 1} | (i,j), (i',j') \in V_2) \cdot P((i,j),(i',j') \in V_2)$$ $$+ P(X_{e_i = 1} \land X_{e_j = 1} | (i,j) \in V_1, (i',j') \in V_2) \cdot P((i,j)\in V_1,(i',j') \in V_2)$$ $$+ P(X_{e_i = 1} \land X_{e_j = 1} | (i,j) \in V_2, (i',j') \in V_1) \cdot P((i,j) \in V_2,(i',j') \in V_1).$$ $$= (1-p)^2 \cdot 1/4 + (1-p)^2 \cdot 1/4 + (1-p)q \cdot 1/4 + (1-p)q \cdot 1/4$$ $$= \frac{(1-p)((1-p) + q)}{2}.$$
So the covariance is equal to $\frac{(1-p)((1-p) + q)}{2}$ plus the (expectation of) the other terms. These values and the variance values give us all entries in the covariance matrix after a single partition/complementing operation. 
\end{document}
