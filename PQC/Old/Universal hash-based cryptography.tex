\documentclass[twocolumn, aps, amsmath, amssymb, nofootinbib, superscriptaddress, longbibliography, doublefloatfix, table-of-contents, eqsecnum, rmp]{revtex4-2}

\usepackage[pdftex]{graphicx}
\usepackage{mathrsfs}
\usepackage[colorlinks, breaklinks, urlcolor={blue}, linkcolor={red}, citecolor={blue}]{hyperref}
\usepackage{amsmath}
\usepackage[english]{babel}
\usepackage{booktabs}
\usepackage{amssymb}
\usepackage{type1cm}
\usepackage{caption}
\usepackage{url}
\usepackage[breaklinks]{hyperref}
    
\frenchspacing
    
\captionsetup[figure]{margin=0pt, font=small, labelfont=bf, labelsep=endash, justification=centerlast, labelsep=colon}
\captionsetup[algorithm]{margin=0pt, font=small, labelfont=bf, labelsep=endash, justification=centerlast, labelsep=colon}
    
\begin{document}

\title{Universal hash-based post-quantum cryptography}

\author{Peter P. Rohde}
%\email[]{peter@peterrohde.org}
%\homepage{https://www.peterrohde.org}
%\affiliation{\mbox{BTQ Technologies, 16-104 555 Burrard Street, Vancouver BC, V7X 1M8 Canada}}
%\affiliation{\mbox{Center for Engineered Quantum Systems, School of Mathematical \& Physical Sciences}, \mbox{Macquarie University, NSW 2109, Australia}}
%\affiliation{Centre of Excellence in Engineered Quantum Systems, Brisbane, Australia}
%\affiliation{\mbox{Hearne Institute for Theoretical Physics, Department of Physics \& Astronomy}, \mbox{Louisiana State University, Baton Rouge LA, United States}}

\begin{abstract}
\end{abstract}

\maketitle

%\tableofcontents

\section{Differential hash codes}

A pair of bit-strings $\{x,y\}$ may be expressed differentially using the tuple,
\begin{align}
	[x,x\oplus y]_\oplus,
\end{align}
where the differential term $x\oplus y$ alone reveals no information about $x$ or $y$ while the non-differential term unlocks the code to reveal both. The validity of differentially encoded tuples may be trivially confirmed given knowledge of both terms.

We define the differential hash operators,
\begin{align}
	\Delta(x) &= h(x)\oplus x,\nonumber\\
	\Delta_\pi(x) &= h(x_\pi)\oplus x,
\end{align}
where $\pi\in S_n$ for $x\in\{0,1\}^n$ is a permutation over the elements of $x$. These encode a hash's image and pre-image together while revealing neither assuming hash pre-image resistance. We have the properties,
\begin{align}
	h(x) &= \Delta(x) \oplus x,\nonumber\\
	x &= \Delta(x) \oplus h(x).
\end{align}
The $\Delta$ operator inherits pre-image resistance from $h(\cdot)$. Knowing $\Delta(x)$ alone reveals neither $x$ nor $h(x)$, however additionally knowing $x$ or $h(x)$ enables verification of $\Delta(x)$. Finding $x$ for given $\Delta(x)$ reduces to the pre-image resistance of the hash function $h(\cdot)$.

The non-differentially encoded tuple $\{x,h(x)\}$ allows $x$ to unlock $h(x)$, while $h(x)$ cannot unlock $x$. The second element reveals $h(x)$ alone, but not $x$ via pre-image resistance. Under the differential encoding,
\begin{align}
	[x,\Delta(x)]_\oplus =[x,h(x)\oplus x]_\oplus,
\end{align}
the second element reveals neither $x$ nor $h(x)$, while the first element reveals both, given that $h(x)$ can be efficiently forward-evaluated. Alternately, under the differential encoding,
\begin{align}
	[h(x),\Delta(x)] = [h(x),h(x)\oplus x],
\end{align}
the non-differential term $h(x)$ affords unlocking the code but does not on its own reveal $x$ via hash pre-image resistance. Under both encodings knowing either $x$ or $h(x)$ alone enables verification.

The differential operator is distributive only over its unhashed components,
\begin{align}
	\Delta(x\oplus y) &= h(x\oplus y)\oplus x\oplus y \nonumber\\
	\Delta(x)\oplus\Delta(y) &= h(x)\oplus h(y)\oplus x\oplus y.
\end{align}
%Thus knowing $h(x\oplus y)$ does not unlock the object, whereas knowing both $x$ and $y$ does.

%Tuples of the form,
%\begin{align}
%	D(x,h(x)) &= [x, \Delta(x)], \nonumber\\
%	D(h(x),x) &= [h(x), \Delta(x)],
%\end{align}
%provide alternate differential encodings for $\{x,h(x)\}$, both affording efficient verification. A convenience is to collect terms known to different parties over the two sides of the encoding. Under both encodings knowing either $x$ or $h(x)$ unlocks both.

% We will generally treat the differential term as public and the non-differential term as private information.

%Taking bit-permuted tuples of this form, hash verification only succeeds if the correct inverse permutation is first applied.

%\subsection{Algebraic properties}

The symmetric difference between $\Delta(x\oplus y)$ and $\Delta(x)\oplus\Delta(y)$ gives the `distributor' (equivalent of commutator for distributivity),
\begin{align}
	\Delta(x\oplus y)\oplus \Delta(x)\oplus \Delta(y) = h(x\oplus y) \oplus h(x)\oplus h(y),
\end{align}
defining the distributivity of $\Delta$ operator over the action of $\oplus$.

Standard differential codes are composable,
\begin{align}
	[x,x\oplus y)]_\oplus \oplus [x',x'\oplus y']_\oplus\nonumber\\
	\sim [x\oplus x', x\oplus y \oplus x' \oplus y']_\oplus.
\end{align}
For differential hash codes,
\begin{align}
	[x,\Delta(x)]_\oplus,\nonumber\\
	[y,\Delta(y)]_\oplus,
\end{align}
we have distinct composition rules,
\begin{align}
	[x\oplus y,\Delta(x\oplus y)]_H,\nonumber\\
	[x\oplus y,\Delta(x) \oplus \Delta(y))]_\oplus.
\end{align}
The $[\cdot,\cdot]_H$ composition is verifiable by hashing the left hand term. The $[\cdot,\cdot]_\oplus$ composition is not hash-verifiable but preserves all differential encoding constraints.

Permutations $\pi$ are distributive over $\oplus$ but not commutative,
\begin{align}
	\pi(x\oplus y) &= \pi(x) \oplus \pi(y),\nonumber\\
	\pi(x) \oplus y &\neq x \oplus \pi(y),
\end{align}
whereas $\oplus$ is commutative but not distributive (in general, depending on parity of number of terms under distribution),
\begin{align}
	x\oplus y &= y \oplus x.
\end{align}

\begin{itemize}
	\item $h(m\oplus s)$ will reveal the private $h(s)$ for chosen $m=\mathbf{0}$.
	\item $h(m\oplus s \oplus x)$ will reveal the private $h(x)$ for chosen $m=x$ if $x$ is public.
	\item $h(\pi(m\oplus s)) = h(m_\pi \oplus s_\pi)$ can only reveal $h(s_\pi)$ (not secret) for public $\pi$ and chosen $m$, but cannot reveal secret $h(s)$.
\end{itemize}

\section{Asymmetric codes}

We define key-pairs as,
\begin{align}
	\mathtt{sk} &\in \{0,1\}^n, \nonumber\\
	\mathtt{pk} &= \{\mathtt{pk}_\Delta,\mathtt{pk}_\pi\},\nonumber\\
	\mathtt{pk}_\Delta &= \Delta(\mathtt{sk}),\nonumber\\
	\mathtt{pk}_\pi &\in S_n,
\end{align}
where $\mathtt{sk}$ be a secret bit-string, $h(\{0,1\}^n)\to\{0,1\}^n$ an $n$-bit endomorphic hash function, and $\pi\in S_n$ a permutation on $n$ bits. Since $|S_n|=n!$ encoding $\pi$ requires $\lceil\log_2(n!)\rceil$ bits. For $n=256$ we have $\lceil\log_2(n!)\rceil = 1684$ bits.

%and,
%\begin{align}
%	\pi &= h(\Delta(\mathtt{sk})),
%\end{align}
%is a non-unlocking public identifier, where,
%\begin{align}
%	x_\pi \equiv x\oplus h(\pi).
%\end{align}
Since $\mathtt{pk} = \Delta(\mathtt{sk})$ is public both $\mathtt{sk}$ and $h(\mathtt{sk})$ must be private to prevent unlocking the public key, both acting as trapdoors for the differential encoding.


%The signature confirms consistency between private and public information while revealing only $h(\mathtt{sk}_pi)$ about the private key. To use Alice's private key as $s$ we inhibit the disclosure of $\mathtt{sk}$ by applying a publicly known pre-image manipulation prevents the signature from revealing the secret key, instead using,

\begin{align}
	[m_\pi \oplus s_\pi, \Delta_\pi(m_\pi\oplus s_\pi)],
\end{align}

\begin{align}
	\Delta_\pi(m_\pi\oplus s_\pi) &= \Delta_\pi(m_\pi) \Delta_\pi(s_\pi) \nonumber\\
	&\oplus h(m_\pi) \oplus h(s_\pi) \oplus h(m_\pi \oplus s_\pi)
\end{align}

%Hence for,
%\begin{align}
%	[\mathtt{sig}, \Delta_\pi(m_\pi) \oplus \Delta(s_\pi)],
%\end{align}
%we require,
%\begin{align}
%	h(\mathtt{sig}) = \Delta_\pi(m_\pi) \oplus \Delta(s_\pi) \oplus h(m_\pi) \oplus h(s_\pi).
%\end{align}


%--
%We will employ the two alternate differential encodings as asymmetric primitives,
%\begin{align}
%	&[m\oplus s, \Delta(h(m)\oplus s)],\nonumber\\
%	&[h(m)\oplus s, \Delta(m\oplus s)],
%\end{align}
%with the identities,
%\begin{align}
%	\Delta(m\oplus s) &= \Delta(m)\oplus \Delta(s) \oplus h(m\oplus s) \oplus h(m)\oplus h(s),\nonumber\\
%	\Delta(h(m)\oplus s) &= \Delta(m)\oplus \Delta(s) \oplus h(h(m)\oplus s) \oplus m\oplus h(s).
%\end{align}
%
%\begin{align} 
%	\Delta(m)\oplus \Delta(s) &= \Delta(m\oplus s) \oplus h(m\oplus s) \oplus h(s) \oplus h(m),\nonumber\\
%\end{align}
%
%\begin{align} 
%	\Delta(m)\oplus \Delta(s) &= \Delta(m\oplus s) \oplus h(m\oplus s) \oplus h(s) \oplus h(m),\nonumber\\
%\end{align}
%
%---
%
%
%\begin{align}
%	\Delta(m_\pi \oplus s) &= h(m_\pi \oplus s) \oplus m_\pi \oplus s,\nonumber\\
%	\Delta(m_\pi) \oplus \Delta(s) &= h(m_\pi) \oplus h(s) \oplus m_\pi \oplus s,
%\end{align}
%
%Assume the message is known,
%\begin{align}
%	\Delta(m_\pi \oplus s_\pi) &= h(m_\pi \oplus s_\pi) \oplus m_\pi \oplus s_\pi \nonumber\\
%	&= \Delta(m_\pi) \oplus \Delta(s) \oplus h(m_\pi \oplus s) \oplus m_\pi \oplus s_\pi.
%\end{align}
%
%\begin{align}
%	\Delta(m_\pi \oplus s_\pi) = \Delta(m_\pi) \oplus \Delta(s) \oplus h(m_\pi \oplus s) \oplus m_\pi \oplus s.
%\end{align}
%
%%\begin{align}
%%	\Delta(s_\pi) &= h(s_\pi) \oplus s_\pi, \nonumber\\
%%	\Delta(s) &= \Delta(s_\pi) \oplus h(s_\pi) \oplus s_\pi.
%%\end{align}
%
%
%Hence (2): given $\Delta(m_\pi)$ Alice can extract the message via,
%\begin{align}
%		m = \Delta(\pi \oplus m) \oplus \Delta(s) \oplus h(\pi \oplus m) \oplus \pi,
%\end{align}
%
%--
%
%\begin{align}
%	&[h(m_\pi \oplus s_\pi), \Delta(m_\pi \oplus s_\pi)],\nonumber\\
%	&[h(m_\pi \oplus s_\pi), \Delta(m_\pi ) \oplus \Delta(s_\pi) \oplus h(m_\pi) \oplus h(s_\pi) \oplus h(m_\pi  \oplus s_\pi))],\nonumber\\
%\end{align}
%
%\begin{align}
%	h(m_\pi \oplus s_\pi) &= \Delta(m_\pi) \oplus \Delta(s_\pi) \nonumber\\
%	&\oplus h(m_\pi) \oplus h(s_\pi) \oplus h(m_\pi \oplus s_\pi),\nonumber\\
%\end{align}

\section{Digital signatures}

To sign message \mbox{$m\in\{0,1\}^n$} Alice makes public the differentially encoded, signed message $\Delta(m_\pi)$ and signature,
\begin{align}
	 \mathtt{sig}_\pi(m) = h(m_\pi \oplus \mathtt{sk}_\pi).
\end{align}
The signature has the property that when combined with public information it reveals the hash of the message being signed,
\begin{align}
	 \mathtt{sig}_\pi(m) \oplus \Delta(m) \oplus \Delta(\mathtt{sk}_\pi) = h(m).
\end{align}
Employing the modulated public key $\Delta(\mathtt{sk}_\pi)$,
\begin{align}
	\mathtt{sk}_\pi \equiv \mathtt{sk} \oplus h(\mathtt{pk}),
\end{align}
prevents the signature from revealing the trapdoor $h(\mathtt{sk})$ which unlocks the public key,
\begin{align}
	h(\mathtt{sk}) \oplus \Delta(\mathtt{sk}) &= \mathtt{sk},\nonumber\\
	h(\mathtt{sk}_\pi) \oplus \Delta(\mathtt{sk}) &\neq \mathtt{sk}.
\end{align}

%To verify Alice's signature Bob evaluates,
%\begin{align}
%	\mathtt{sig}_\pi(m) \oplus \Delta(s_\pi) \oplus \Delta(m) = m,
%\end{align}
%revealing $m$, which can be compared against its hash to verify the signature's validity.

%Taking Eq.~\eqref{eq:diff_sol} and grouping public information on the right we obtain,
%\begin{align}
%	h(m\oplus s) &= \Delta(m)\oplus \Delta(s),\nonumber\\
%	\mathtt{sig} &= h(m\oplus s).
%\end{align}

\section{Encryption}

For asymmetric encryption we reverse the roles of $m$ and $h(m)$. Bob wishes to send message $m$ to Alice and makes public,
\begin{align}
	\mathtt{enc}(m) &= \{\Delta(m), h(m)\}. 
\end{align}
Alice now decrypts using the message hash $h(m)$ to reveal the original message,
\begin{align}
	\Delta(s_\pi) \oplus \Delta(m) \oplus h(m) = m.
\end{align}

\subsection{Multi-sigs}

Signatures are commutative, additive and composable.
\begin{align}
	\mathtt{sig}_\pi(m) = \Delta(s_\pi) \oplus h(m),
\end{align}

\subsection{Key-establishment \& secret sharing}

Using the asymmetric encryption protocol, Alice finally communicates the verification hash,
\begin{align}
	\mathtt{key} = h(\mathtt{sk}\oplus m),
\end{align}
back to Bob, who also able to verify its validity. The verification hash now provides confirmation of a jointly prepared hash-based random number given by  the XOR-salted hash of Alice's secret key and Bob's chosen salt, which cannot be spoofed by either.

\section{Notes}

* The hash collision space translates to decoding failure.

Differentially encoded tuples are composable under bitwise XOR,
\begin{align}
	[x,\Delta(x)] \oplus [y,\Delta(y)] = [x\oplus y, \Delta(x)\oplus \Delta(y)],
\end{align}
via the commutativity of $\oplus$.

Compare above rhs,
\begin{align}
	h(x\oplus y) \oplus x \oplus y = h(x \oplus y) \oplus h(x) \oplus h(y).
\end{align}

$\pi$ known by inverting $\pi$ and multiplying the tuple elements to confirm consistency. For unknown or incorrect $\pi$ hash verification fails.

The bit permutation operator, $\pi(\cdot)$, is distribute over the bit-wise XOR operator, $\oplus$,
\begin{align}
	\pi(x)\oplus \pi(y) = \pi(x\oplus y).
\end{align}
Hence differential encoding relationships are preserved under the uniform action of $\pi$,
\begin{align}
	[x,x\oplus y] &\sim [\pi(x), \pi(x\oplus y)],
\end{align}
but are in general not preserved under non-uniform action of $\pi$,
\begin{align}
	[x,x\oplus y] \not\sim [\pi(x), x\oplus y].
\end{align}
We'll employ the shorthand,
\begin{align}
	\pi\circ[x,y] = [\pi(x),\pi(y)].,
\end{align}
to denote the uniform action of $\pi$ over a differential code.

While permutations are distributive over $\oplus$ they do not commute through hashes,
\begin{align}
	\pi(h(x)) \neq h(\pi(x)).	
\end{align}

\begin{align}
	\Delta(\pi(x\oplus y)) = h(\pi(x\oplus y)) \oplus \pi(x) \oplus \pi(y).
\end{align}

\bibliography{bibliography}

\end{document}

%\subsubsection{Notes}
%
%If $\Delta(x)=h(x)\oplus x$ is public information both $x$ and $h(x)$ must be private.
%
%Decrypt $m$ with $h(m)$ requires the locking construction,
%\begin{align}
%	&[h(m), m\oplus h(m)]_\oplus	= [h(m), \Delta(m)]_\oplus,\nonumber\\
%	&[s\oplus h(m), \Delta(s) \oplus \Delta(m)]_\oplus,\nonumber\\
%	&[s\oplus h(m), \Delta(s\oplus h(m))]_\oplus \nonumber\\
%	&[s\oplus h(m), \Delta(s) \oplus \Delta(m) \oplus h(s\oplus h(m)) \oplus h(m) \oplus h(s)]_\oplus,\nonumber\\
%%	&[s\oplus h(m), \Delta(s) \oplus \Delta(m) \oplus  \oplus h(m) \oplus h(s)]_\oplus
%\end{align}
%if $\Delta(m)$ is public.
%
%
%This does not unlock from known $\pi$,
%\begin{align}
%	[h(\pi(m)), \pi(m) \oplus h(m)]_\oplus
%\end{align}
%
%This unlocks,
%\begin{align}
%	[h(\pi(s \oplus m)), \Delta(s \oplus m)]_\oplus
%\end{align}
%
%%Equivalently,
%%\begin{align}
%%	&[h(s\oplus h(m)), \Delta(s\oplus h(m))]_\oplus \nonumber\\
%%	&[s \oplus h(m), \Delta(s) \oplus \Delta(m)]_\oplus \nonumber\\
%%	&[\pi(s) \oplus h(m), \Delta(s) \oplus \Delta(m)]_\oplus \to m \nonumber\\
%%	&[\pi(s) \oplus \pi(h(m)), \Delta(s) \oplus \pi(\Delta(m))]_\oplus \to m \nonumber\\
%%\end{align}
%%unlocks for known $\Delta(\pi(s))$ (private).
%
%Want to decrypt with,
%\begin{align}
%	h(\pi(s\oplus h(m)) &= \Delta(\pi(s\oplus h(m))) \nonumber\\
%	&\oplus \pi(s) \oplus \pi(h(m)).
%\end{align}
%
%\begin{align}
%	[h(m), \Delta(\pi(m))\oplus \Delta(\pi(s))]_\oplus
%\end{align}
%unlocks for known $\Delta(\pi(s))$ (private).
%
%\begin{align}
%	\Delta(\pi(m))\oplus \Delta(\pi(s))	= \Delta
%\end{align}
%
%
%\begin{align}
%	&\mathtt{sig} = \Delta(m) = \pi(m) \oplus h(\pi(m)) \nonumber\\
%	&\pi(m) = \mathtt{sig} \oplus h(\pi(m)) 
%\end{align}
%
%\subsection{Digital signatures}
%
%Consider parties Alice and Bob where Alice wants to sign the message \mbox{$m\in \{0,1\}^n$}. Alice prepares the signature,
%\begin{align}
%	\mathtt{sig}(m) &= \{\pi(m), \Delta(m), h(\mathtt{sk} \oplus m)\}.
%\end{align}
%
%Bob prepares $\pi(\Delta(m))$ and knows $\pi(\Delta(\mathtt{sk}))$. We have the relationship,
%\begin{align}
%	\pi(\mathtt{sk} \oplus m) = \pi(\Delta(\mathtt{sk}) \oplus \Delta(m) \oplus h(m) \oplus(h(\mathtt{sk}))%  \oplus h(\mathtt{sk}) \oplus h(m)
%\end{align}
%
%The encodings,
%\begin{align}
%	[\mathtt{sk}, \Delta(\mathtt{sk})]_H, [h(\mathtt{sk}), \Delta(\mathtt{sk})]_\oplus \nonumber\\
%	[m, \Delta(m)]_H, [h(m), \Delta(m)]_\oplus \nonumber\\
%	[m \oplus \mathtt{sk} \oplus h(m) \oplus h(\mathtt{sk}), \Delta(m) \oplus \Delta(\mathtt{sk})]_\oplus
%\end{align}
%enable verification by hashing the first term.
%
%The differential code,
%\begin{align}
%	[\mathtt{sk} \oplus m, \Delta(\mathtt{sk} \oplus m)] %\oplus h(m) \oplus h(\mathtt{sk})]
%\end{align}
%is verifiable given $h(\mathtt{sk} \oplus m)$, while the permuted code,
%\begin{align}
%	[\pi(\mathtt{sk} \oplus m), \Delta(\pi(\mathtt{sk} \oplus m))]_H 
%\end{align}
%is only verifiable with $h(\pi(\mathtt{sk} \oplus m))$.
%
%\begin{align}
%	&[h(\mathtt{sk} \oplus m), \Delta(\mathtt{sk} \oplus m)]_\oplus \nonumber\\
%	&[h(\pi(\mathtt{sk} \oplus m)), \Delta(\pi(\mathtt{sk} \oplus m))]_\oplus
%\end{align}
%are both verifiable from direct multiplication.
%
%%\begin{align}
%%	&[h(\pi(\mathtt{sk} \oplus m)), \Delta(\mathtt{sk} \oplus m)]_\oplus \nonumber\\
%%	&[h(\pi(\mathtt{sk} \oplus m)), h(\mathtt{sk} \oplus m)\oplus \mathtt{sk} \oplus m]_\oplus
%%\end{align}
%%is verifiable from direct multiplication where $\pi$ is known.
%
%
%
%%\begin{align}
%%	\pi(\Delta(\mathtt{sk} \oplus m)) &= \pi(h(\mathtt{sk} \oplus m)) \oplus \pi(\mathtt{sk}) \oplus \pi(\Delta(m)\oplus h(m)).
%%\end{align}
%
%\subsection{Blah}
%
%Hence,
%\begin{align}
%	[h(\pi(\mathtt{sk} \oplus m)), \Delta(\pi(\mathtt{sk} \oplus m))]
%\end{align}
%affords direct verification by multiplying ($t_1\cdot t_2 = h(t_2)$). The second term is equivalent to,
%\begin{align}
%	t_2 &= \Delta(\pi(\mathtt{sk} \oplus m)) \nonumber\\
%	&= \Delta(\pi(\mathtt{sk})) \oplus \Delta(\pi(m)) \nonumber\\
%	&\oplus h(\pi(\mathtt{sk} \oplus m))
% \oplus h(\pi(\mathtt{sk})) \oplus h(\pi(m))
%\end{align}
%
%%\begin{align}
%%	\pi(\Delta(\pi(x))) &= \pi(h(\pi(x))) \oplus x	
%%\end{align}
%
%Let signature be,
%\begin{align}
%	\mathtt{sig} &= h(\pi(\mathtt{sk} \oplus m)),\nonumber\\
%\end{align}
%
%\begin{align}
%	\mathtt{ver} &= \Delta(\pi(\mathtt{sk} \oplus m)) \nonumber\\
%\end{align}
%
%\begin{align}
%	[\mathtt{sig}, \mathtt{ver}]
%\end{align}
%
%
%\section{Blah}
%
%\begin{align}
%	\Delta(\pi(\mathtt{sk} \oplus m)) &= h(\pi(\mathtt{sk} \oplus m)) \oplus \pi(\mathtt{sk} \oplus m) \nonumber\\
%	&= h(\pi(\mathtt{sk}) \oplus \pi(m)) \oplus \pi(\mathtt{sk} \oplus m). \nonumber\\
%	\Delta(\pi(\mathtt{sk})) \oplus \Delta(\pi(m)) &= \pi(\mathtt{sk}) \oplus \pi(m) \nonumber\\
%	&\oplus h(\pi(\mathtt{sk})) \oplus h(\pi(m)).\nonumber\\
%	\Delta(\pi(\mathtt{sk} \oplus m)) &= \Delta(\pi(\mathtt{sk})) \oplus \Delta(\pi(m)) \nonumber\\
%	&\oplus h(\pi(\mathtt{sk} \oplus m))
% \oplus h(\pi(\mathtt{sk})) \oplus h(\pi(m))
%\end{align}
%
%\begin{align}
%	\Delta(x\oplus\pi(x)) &= h(x\oplus \pi(x)) \oplus x \oplus \pi(x).
%\end{align}
%
%\begin{align}
%	\Delta(\mathtt{sk} \oplus m \oplus \pi(\mathtt{sk}\oplus m)) &= h(\mathtt{sk} \oplus m \oplus \pi(\mathtt{sk}) \oplus \pi(m)) \nonumber\\
%	&\mathtt{sk} \oplus m \oplus \pi(\mathtt{sk}) \oplus \pi(m).
%\end{align}
%
%The differential permuted code,
%\begin{align}
%	&[\pi(\mathtt{sk} \oplus m), \Delta(\pi(\mathtt{sk} \oplus m))]
%\end{align}
%is verifiable from,
%\begin{align}
%	h(\pi(\mathtt{sk} \oplus m))
%\end{align}
%
%The differential permuted code,
%\begin{align}
%	&[\pi(\mathtt{sk} \oplus m), \Delta(\pi(\mathtt{sk} \oplus m))]
%\end{align}
%is verifiable from,
%\begin{align}
%	&h(\pi(\mathtt{sk} \oplus m)) \oplus \pi(\mathtt{sk} \oplus m) = bob \nonumber\\
%	&\oplus h(\pi(\mathtt{sk} \oplus m)) \oplus h(\pi(\mathtt{sk})) \oplus h(\pi(m)) \nonumber\\
%	&= 
%\end{align} 
%
%While,
%\begin{align}
%	&[\mathtt{sk} \oplus m, \Delta(\mathtt{sk} \oplus m)] \nonumber\\
%	&= [\mathtt{sk} \oplus m, \Delta(\mathtt{sk}) \oplus \Delta(m) \nonumber\\
%	&\oplus h(\mathtt{sk} \oplus m) \oplus h(\mathtt{sk}) \oplus h(m)],
%\end{align}
%is verifiable from $h(\mathtt{sk} \oplus m)$.
%
%Bob's test passes if,
%\begin{align}
%	\pi(\mathtt{sk} \oplus m) &= \Delta(\pi(\mathtt{sk})) \oplus \Delta(\pi(m)) \oplus h(\pi(\mathtt{sk})) \oplus h(\pi(m))
%\end{align}
%
%
%
%Bob is not allowed to know $h(\mathtt{sk})$.
%
%Similarly,
%\begin{align}
%	&[\mathtt{sk} \oplus m, \Delta(\mathtt{sk}\oplus m)] \nonumber\\
%	&= [\mathtt{sk} \oplus m, h(\mathtt{sk}\oplus m) \oplus h(\mathtt{sk}) \oplus h(m)],
%\end{align}
%are verifiable from hashing the first term, while,
%\begin{align}
%	&[\pi(\mathtt{sk} \oplus m), \Delta(\mathtt{sk}\oplus m)] \nonumber\\
%	&= [\pi(\mathtt{sk} \oplus m), h(\mathtt{sk}\oplus m) \oplus h(\mathtt{sk}) \oplus h(m)],
%\end{align}
%are verifiable by hashing the unpermuted first term. Equivalently,
%\begin{align}
%	&[\pi(\mathtt{sk} \oplus m), \Delta(\mathtt{sk}\oplus m)] \nonumber\\
%	&[\pi(\mathtt{sk} \oplus m), \Delta(\mathtt{sk})\oplus \Delta(m) \oplus h(\mathtt{sk}\oplus m) \oplus h(\mathtt{sk}) \oplus h(m)] \nonumber\\\end{align}
%is verifiable by hashing the unpermuted first term.
%
%Using,
%\begin{align}
%	\Delta(\mathtt{sk}) \oplus \Delta(m) &= \Delta(\mathtt{sk} \oplus m) \oplus h(\mathtt{sk} \oplus m)	\oplus h(\mathtt{sk}) \oplus h(m),\nonumber\\
%\end{align}
%
%%We have,
%%\begin{align}
%%	\pi(\Delta(\mathtt{sk}\oplus m)) = \pi(\Delta(\mathtt{sk}) \oplus \Delta(m) h(\mathtt{sk}\oplus m) \oplus h(\mathtt{sk}) \oplus h(m))
%%\end{align}
%
%Ver:
%\begin{align}
%	\{\pi(m)\oplus \pi(\mathtt{sk}), \Delta(m)\oplus \Delta(\mathtt{sk})\} \nonumber\\
%	\{\pi(m\oplus \mathtt{sk}), \Delta(m)\oplus \Delta(\mathtt{sk})\}
%\end{align}
%
%We have the relationships,
%\begin{align}
%	\Delta(\mathtt{sk}) \oplus \Delta(m) &= \mathtt{sk} \oplus m \oplus h(\texttt{sk}) \oplus h(m),\nonumber\\
%	\Delta(\mathtt{sk} \oplus m)	 &= h(\mathtt{sk} \oplus m) \oplus \mathtt{sk} \oplus m,\nonumber\\
%\end{align}
%Under composition we obtain,
%\begin{align}
%	\Delta(\mathtt{sk} \oplus m)	 \oplus \Delta(\mathtt{sk}) \oplus \Delta(m) &= h(\mathtt{sk} \oplus m) \oplus h(\mathtt{sk}) \oplus h(m).	
%\end{align}
%The differential encoding,
%\begin{align}
%	[h(\mathtt{sk}\oplus m), \Delta(\mathtt{sk} \oplus m)],
%\end{align}
%
%Consider the permuted differential code,
%\begin{align}
%	[\pi(\mathtt{sk}\oplus m), \pi(\Delta(\mathtt{sk})) \oplus \pi(\Delta(m))].
%\end{align}
%we equivalently obtain,
%\begin{align}
%	[\Delta(\mathtt{sk} \oplus m) \oplus h(\mathtt{sk} \oplus m), \Delta(\mathtt{sk}) \oplus \Delta(m)].
%\end{align}
%%is verifiable if $\pi(\mathtt{sk} \oplus m)$ is known.
%
%Therefore if Alice provides the signature,
%\begin{align}
%	\mathtt{sig}(m) = [\pi(\mathtt{sk} \oplus m), h(\mathtt{sk} \oplus m)],
%\end{align}	
%Bob is able to verify via,
%\begin{align}
%		[\mathtt{sig}(m),\Delta(\mathtt{sk}) \oplus \Delta(m)].
%\end{align}
%
%
%%where $\Delta(m)$ are $\Delta(\mathtt{sk})$ are known to Bob.
%
%%which Bob is able to verify using Alice's verification hash,
%%\begin{align}
%%	h(\mathtt{sk}\oplus m).
%%\end{align}
%%
%%For $n$-bit messages 
