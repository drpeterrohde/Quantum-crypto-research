\begin{tikzpicture}[genericStyle]
  % Radius of the circle
  \def\radius{5em}
  \def\numnodes{8}
  \def\degree{360/\numnodes}

  % Draw dummy nodes
  \foreach \x in {1,...,\numnodes} {
      \node[draw=none] (node\x) at (\x*\degree+0.5*\degree+180:\radius) {};
    }

  % Draw red lines
  \foreach \x in {1,...,\numnodes} {
      \foreach \y in {\x,...,\numnodes} {
          \pgfmathparse{(\y>\x+1) && (1-(\x==1 && \y==\numnodes)) ? 1 : 0}
          \ifnum \pgfmathresult=1
            \draw[graphNonEdgeStyle, line width=1] (node\x) -- (node\y);
          \fi
        }
    }

  \node[draw, circle, minimum size=2*\radius, graphNonEdgeStyle, line width=1] at (0,0) {};

  % Visible nodes
  \foreach \x in {1,...,\numnodes} {
      \node[nodeGrayStyleC] (node\x) at (\x*\degree+0.5*\degree+180:\radius) {};
    }

  \draw[-{Latex[scale=0.5]}, graphBlueEdgeStyle, line width=2] (node1) -- (node2);
  \draw[-{Latex[scale=0.5]}, graphBlueEdgeStyle, line width=2] (node1) -- (node3);
  \draw[-{Latex[scale=0.5]}, gray!80, line width=2] (node1) -- (node4);
  \draw[-{Latex[scale=0.5]}, graphBlueEdgeStyle, line width=2] (node1) -- (node5);
  \draw[-{Latex[scale=0.5]}, graphRedEdgeStyle, line width=2] (node1) -- (node6);
  \draw[-{Latex[scale=0.5]}, gray!80, line width=2] (node1) -- (node7);
  \draw[-{Latex[scale=0.5]}, graphRedEdgeStyle, line width=2] (node1) -- (node8);

  \node[draw=none] at (node1) {$i$};
  \node[draw=none] at (node2) {$j$};
  \node[draw=none] at ($(node1)!0.5!(node2) + (0,-0.7)$) {$v_{i,j}=\{0,1,\bot\}$};

\end{tikzpicture}