Blockchains rely on distributed consensus algorithms to decide whether proposed transactions are valid and should be added to the blockchain. The purpose of consensus is to act as an independent arbiter for transactions, robust against adversarial manipulation. This can be achieved by choosing random subsets of nodes to form consensus sets. In an economy where consensus is the commodity, consensus must be secure, computationally efficient, fast and cheap. Most current blockchains operate in the context of open networks, where algorithms such as proof-of-work are highly inefficient and resource-intensive, presenting long-term scalability issues. Inefficient solutions to allocating consensus sets equates to high transaction costs and slow transaction times. Closed networks of known nodes afford more efficient and robust solutions. We describe a secure distributed algorithm for solving the random subset problem in networks of known nodes, bidding to participate in consensus for which they are rewarded, where the randomness of set allocation cannot be compromised unless all nodes collude. Unlike proof-of-work, the algorithm allocates all nodes to consensus sets, ensuring full resource utilisation, and is highly efficient. The protocol follows self-enforcing rules where adversarial behaviour only results in self-exclusion. Signature-sets produced by the protocol act as timestamped, cryptographic proofs-of-consensus, a commodity with market-determined value. The protocol is highly strategically robust against collusive adversaries, affording subnetworks defence against denial-of-service and majority takeover.
